% -*- program: xelatex -*-
\documentclass[a4paper,12pt,openany]{book}
\usepackage[slantfont,boldfont]{xeCJK}
\usepackage{CJKnumb}
%\usepackage{fontspec,xltxtra,xunicode}
%\usepackage[slantfont,boldfont]{xeCJK}
\setCJKmainfont[BoldFont={Songti SC}, ItalicFont={KaiTi}]{SimSun}
\setmainfont{Times New Roman}

\makeatletter
\def\CTEX@section@format{\Large\bfseries}
\makeatother


\usepackage{amsmath}
\usepackage{amsfonts}
%\usepackage{ccmap}
\usepackage{setspace}
\usepackage{booktabs}
\usepackage[top=2.54cm,bottom=2.54cm,left=3.17cm,right=3.17cm,includehead,includefoot]{geometry} %% 页面设置
\usepackage[section]{placeins} %% 要求浮动的图形在它们所在的章节中排出选项,如section改为below,则放松要求到下一个section开始
\usepackage[figuresright]{rotating} %旋转表格方向
\usepackage{enumerate}
\usepackage{graphicx} %% 图形支持
\usepackage{subfigure}
\usepackage{indentfirst} %% 首行缩进宏包
\usepackage{cite} %% 支持引用的宏包
%\usepackage{bibspacing}
%\setlength{\bibspacing}{\baselineskip}

\usepackage{bm}
\usepackage{multirow}
\usepackage{amsmath,amssymb}
\usepackage[chapter]{algorithm}
\usepackage{algorithmic}
\usepackage{listings}

%% Begin 字号设置=================================
\newcommand{\chuhao}{\fontsize{42pt}{\baselineskip}\selectfont} %% 初号
\newcommand{\xiaochuhao}{\fontsize{36pt}{\baselineskip}\selectfont} %% 小初号
\newcommand{\yihao}{\fontsize{28pt}{\baselineskip}\selectfont} %% 一号
\newcommand{\erhao}{\fontsize{22pt}{\baselineskip}\selectfont} %% 二号
\newcommand{\xiaoerhao}{\fontsize{18pt}{\baselineskip}\selectfont} %% 小二号
\newcommand{\sanhao}{\fontsize{15.75pt}{\baselineskip}\selectfont} %% 三号
\newcommand{\xiaosanhao}{\fontsize{15pt}{\baselineskip}\selectfont} %% 小三号
\newcommand{\sihao}{\fontsize{14pt}{\baselineskip}\selectfont} %% 四号
\newcommand{\xiaosihao}{\fontsize{12pt}{\baselineskip}\selectfont} %% 小四号
\newcommand{\wuhao}{\fontsize{10.5pt}{\baselineskip}\selectfont} %% 五号
\newcommand{\xiaowuhao}{\fontsize{9pt}{\baselineskip}\selectfont} %% 小五号
\newcommand{\liuhao}{\fontsize{7.875pt}{\baselineskip}\selectfont} %% 六号
\newcommand{\qihao}{\fontsize{5.25pt}{\baselineskip}\selectfont} %% 七号
%% End 字号设置=================================

%% test =============================
%\usepackage[perpage,symbol]{footmisc}   % 脚注控制
%\usepackage{type1cm}
%\usepackage{palatino}
%\usepackage{listings}  % 源代码列表
%\usepackage{wasysym}
%\usepackage{mathrsfs}   % 不同于\mathcal or \mathfrak 之类的英文花体字体
%\usepackage[sort&compress,numbers]{natbib}  % 支持引用的宏包
%\usepackage{subfigure}
%\usepackage{floatflt}
%\usepackage{caption2}
%\usepackage{tweaklist} % 调整列表环境
%\usepackage{algorithm, algpseudocode}
\renewcommand{\today}{\number\year 年 \number\month 月 \number\day 日}
\setlength{\baselineskip}{10pt plus2pt minus2pt} %% 设置行距为20pt
%\setlength{\parskip}{4pt plus2pt minus2pt} %% 段落间距 为8 +2 -2之间浮动
\setlength{\parskip}{8pt plus2pt minus2pt} %% 段落间距 为8 +2 -2之间浮动
\renewcommand{\baselinestretch}{1}
\setlength{\parskip}{0.5\baselineskip}
\usepackage{fancyhdr}
\usepackage[sf]{titlesec}
\titleformat{\chapter}[hang]{\normalfont\LARGE\filcenter}{\bf 第~\CJKnumber{\thechapter}~章}{20pt}{\LARGE}
\titlespacing{\chapter}{0pt}{-3ex  plus .1ex minus .2ex}{2.5ex plus .1ex minus .2ex}
\titleformat{\section}[hang]{\bf \Large }{\Large \ \thesection}{1em}{}{}
\titlespacing{\section}{0pt}{1.5ex plus .1ex minus .2ex}{\wordsep}
\titleformat{\subsection}[hang]{\bf\large}{\large \ \thesubsection}{1em}{}{}
\titlespacing{\subsection}{0pt}{1.5ex plus .1ex minus .2ex}{\wordsep}
\titleformat{\subsubsection}[hang]{\bf }{ \ \thesubsubsection}{1em}{}{}
\titlespacing{\subsubsection}{0pt}{1.5ex plus .1ex minus .2ex}{\wordsep}

\pagestyle{fancy}
\renewcommand{\chaptermark}[1]{\markboth{第\CJKnumber\thechapter 章\ #1}{}}
%以上指令中\chaptermark用于重新定义页眉内之章标题;
%其内容取自LaTeX的\markboth指令。事实上\markboth指令用于存放两项信息,分别
%存放于指令之后的两个大括号内。在book文件类别中,第一项为章名的相关信息;
%第二项为节标题的相关信息。故以上命令是重新定义章标题存放于页眉/页脚的内容,
%节标题不变,因此第二对大括号内为空白
\fancyhf{}
\fancyhead[LE,RO]{\thepage} %% 注:在book文件类别下,\leftmark自动存录各章之章名,\rightmark记录节标题
\fancyhead[LO]{\CJKfamily{fs}复旦大学博士学位论文}
\fancyhead[RE]{\CJKfamily{fs}\leftmark}
%% test =============================

\usepackage[colorlinks,citecolor=green,linkcolor=blue,CJKbookmarks=true]{hyperref}%% 链接支持,需放在定理环境之前,fancyhdr之后,防止冲突,书签有乱码
%\usepackage[dvipdfm,colorlinks,citecolor=green,linkcolor=blue,CJKbookmarks=true]{hyperref} %% 链接支持,需放在定理环境之前,fancyhdr之后,防止冲突,书签无乱码,需要TeXify->LaTeX->dvi2pdf
\usepackage{theorem}
%\usepackage[latin1]{inputenc}
\usepackage{tikz}
\usetikzlibrary{arrows,shapes,positioning,shadows,trees}
%\usegdlibrary{trees}

%% Begin 自定义 定理环境=================================




\usepackage{enumitem}
%\let\OLDitemize\itemize
%\renewcommand\itemize{\OLDitemize\addtolength{\itemsep}{14pt}}
%\let\tempone\itemize
%\let\temptwo\enditemize
%\renewenvironment{itemize}{\tempone\addtolength{\itemsep}{0.5\baselineskip}}{\tempone}
%\renewenvironment{enumerate}{\tempone\addtolength{\itemsep}{0.5\baselineskip}}{\temptwo}
%% End 自定义 定理环境=================================
\renewcommand{\figurename}{图}
\renewcommand{\tablename}{表}
\renewcommand{\bibname}{参考文献}
\renewcommand{\contentsname}{目录}
\renewcommand{\listfigurename}{插图索引}
\renewcommand{\listtablename}{表格索引}
\renewcommand{\listalgorithmname}{算法索引}
\floatname{algorithm}{算法}
\hypersetup{CJKbookmarks=true}
\newcommand{\tabincell}[2]{\begin{tabular}{@{}#1@{}}#2\end{tabular}}

%\newtheorem{proof}{\textbf{证明}:}
\newenvironment{proof}{\paragraph{\emph{\bf 证明:}}}{\hfill$\square$}
\newtheorem{definition}{\textbf{定义}}[chapter] %% 按照section编号

\newtheorem{theorem}[definition]{{\bf 定理}} %% 与definition统一编号

 %% 与definition统一编号
%\newtheorem{lemma}{Lemma}

\newtheorem{proposition}[definition]{\textbf{命题}}
\newtheorem{lemma}[definition]{\textbf{引理}} %% 与definition统一编号
\newtheorem{corollary}[definition]{\textbf{推论}} %% 与definition统一编号
\newtheorem{condition}{\textbf{条件}}[chapter]
\newtheorem{property}{\textbf{性质}}[chapter]
\newcommand{\pozhehao}{\kern0.3ex\rule[0.8ex]{2em}{0.1ex}\kern0.3ex}
%\usepackage{tocloft}

\newcommand{\ie}{{i.e.}}
\newcommand{\eg}{{e.g.}}
\newcommand{\etal}{\emph{et al.}}
%% figures paths

%\graphicspath{{chapter1/}{chapter3/}{chapter4/}}

\begin{document}


\frontmatter
\pagenumbering{Roman}

%\input{00.00.Cover_blind.tex}
% -*- program: xelatex -*-
% !TEX root =  main.tex
%\documentclass{article}
%\usepackage{CJK}
%\usepackage{graphicx}
\thispagestyle{empty}

%\vspace*{2mm}
\hspace{10.2cm}{\xiaowuhao  学校代码:10246}\\
\vspace*{2mm}
\hspace{10.7cm}{\xiaowuhao 学\quad\quad 号:13110240011
}\\

\vspace*{8mm}
\begin{center}
\includegraphics[width=9cm]{images/xiaoming.eps}
\end{center}
\vspace{10mm}
 \centerline{{\erhao 博\ \ \ 士\ \ \ 学\ \ \ 位\ \ \ 论\ \ \ 文}}
  \vspace{5mm}
 \centerline{{\bf \xiaosanhao (学术学位)}}
 \vspace{9mm}
 \vspace{15mm}
 \begin{center}
 {\bf \xiaoerhao 博士论文中文题目}\\
 \end{center}
 \vspace{4mm}
  \begin{center}
 {\bf \sihao English title}\\
 \end{center}

 \vspace{1.7cm}
 %\large
 \begin{tabbing}
 \hspace*{3.5cm} \= \hspace{2.6cm} \= \kill

 \>{\sihao 院~ ~ ~系:}\>\hspace{0.5cm}{\sihao 计算机科学技术学院}\\
 \\
 \>{\sihao 专~ ~ ~业:}\>\hspace{0.5cm}{\sihao 计算机软件与理论}\\
 \\
 \>{\sihao 姓~ ~ ~名:}\>\hspace{0.5cm}{\sihao 张三
 }\\ %总长3.4cm,即
 \\
 \>{\sihao 指~导~教~师:}\>\hspace{0.5cm}{\sihao 李四 教授
 }\\
 \\
 \>{\sihao 完~成~日~期:}\>\hspace{0.5cm}{\sihao \today}\\

 \end{tabbing}
 \normalsize


 \newpage
 \thispagestyle{empty}

 \vspace*{8mm} %插入空白
 \begin{center}

 {\Large 指导小组成员名单}\\

 \vspace{3cm}

\hspace{4cm} {\Large 某某~~~~~~教授} \hspace{1cm}  {中国,复旦大学  }\\ %总长3.4cm


%总长3.4cm

 \end{center}

 \normalsize


%\end{CJK*}
%\end{document}

\cleardoublepage
% -*- program: xelatex -*-
% !TEX root =  main.tex
\chapter*{摘要}
\addcontentsline{toc}{chapter}{摘要}

你的摘要内容

{\textbf{关键词:关键词1,关键词2,关键词3,}}

{\textbf{中图分类号:TP181}}
\addtocontents{toc}{\protect\vskip-10pt}
\cleardoublepage
% -*- program: xelatex -*-
% !TEX root =  main.tex
\chapter*{Abstract}
\addcontentsline{toc}{chapter}{Abstract}
Abstract

{\textbf{Keywords: Keyword1, keyword2, keyword 3}}

{\textbf{Chinese Library Classification: TP181}}\addtocontents{toc}{\protect\vskip-10pt}
\cleardoublepage
% -*- program: xelatex -*-
% !TEX root =  main.tex
\chapter*{主要符号表}
\addcontentsline{toc}{chapter}{主要符号表}

\begin{table}[h]
\centering
	\begin{tabular}{lcl}
	$x$ 		& 	&	标量 		\\ 
	
	$X$ 		&  	& 	随机变量 		\\
	
	\end{tabular}
\end{table}\addtocontents{toc}{\protect\vskip-10pt}
\cleardoublepage


%\setlength\cftparskip{-2pt}
%\setlength\cftbeforechapskip{0pt}
\setcounter{secnumdepth}{2}
%\clearpage
\addcontentsline{toc}{chapter}{\contentsname}
\setlength{\baselineskip}{14pt plus2pt minus2pt}
\addtocontents{toc}{\protect\vskip-10pt}
\tableofcontents
\cleardoublepage

%\clearpage
\addcontentsline{toc}{chapter}{\listfigurename}
\listoffigures
\addtocontents{toc}{\protect\vskip-10pt}
\cleardoublepage


{\addcontentsline{toc}{chapter}{\listtablename}
\listoftables}
\addtocontents{toc}{\protect\vskip-10pt}
\cleardoublepage


\setlength{\baselineskip}{20pt plus2pt minus2pt}
\mainmatter
%\input{00.01.preface}\addtocontents{toc}{\protect\vskip-10pt}\cleardoublepage
%\listofalgorithms
%\setlength{\baselineskip}{14pt plus2pt minus2pt}




% -*- program: xelatex -*-
% !TEX root =  main.tex
\chapter{绪论}

你博士期间的研究背景

\begin{figure}
	\centering
	\includegraphics[width=1\linewidth]{IMAGES/xiaoming.eps}
	\caption{复旦大学校名LOGO}
\end{figure}

\begin{table}
	\centering
	\begin{tabular}{c|c}
	A & B \\ \hline
	C & D \\
	\end{tabular}
	\caption{表格}
\end{table}\addtocontents{toc}{\protect\vskip-10pt}\cleardoublepage
% -*- program: xelatex -*-
% !TEX root =  main.tex
\chapter{第二章}

说点啥呢?



\addtocontents{toc}{\protect\vskip-10pt}\cleardoublepage
% -*- program: xelatex -*-
% !TEX root =  main.tex

\chapter{第三章}
再说点啥




\addtocontents{toc}{\protect\vskip-10pt}\cleardoublepage
% -*- program: xelatex -*-
% !TEX root =  main.tex


\chapter{第四章}
最后说点啥



\addtocontents{toc}{\protect\vskip-10pt}\cleardoublepage
% -*- program: xelatex -*-
% !TEX root =  main.tex
\chapter{总结与展望}

总结一下博士期间完成的,再说点还没完成的。。



\addtocontents{toc}{\protect\vskip-10pt}\cleardoublepage
%\input{05.00.Appendix}
\backmatter

\clearpage
\phantomsection
\addcontentsline{toc}{chapter}{\bibname} % 解决目录中参考文献的问题
\bibliographystyle{gbt7714-2005}

\setlength{\baselineskip}{14pt plus2pt minus2pt}
\bibliography{reference}
\cleardoublepage
%\setlength{\baselineskip}{20pt plus2pt minus2pt}
\addtocontents{toc}{\protect\vskip-10pt}
% -*- program: xelatex -*-
% !TEX root =  main.tex
\chapter*{科研成果}
\fancyhead[R]{科研成果}%右页眉设置
\addcontentsline{toc}{chapter}{科研成果}

不写点怎么毕业@@\cleardoublepage
% -*- program: xelatex -*-
% !TEX root = ./main.tex
\chapter*{致谢}
\fancyhead[R]{致谢}
\addcontentsline{toc}{chapter}{致谢}

谢老师,谢父母,\cleardoublepage \addtocontents{toc}{\protect\vskip-10pt}

% -*- program: xelatex -*-
% !TEX root =  main.tex
\thispagestyle{empty}
%\renewcommand{\baselinestretch}{1.6}
%\fontsize{12pt}{13pt}\selectfont
%\setlength{\baselineskip}{20pt plus2pt minus2pt}

 %\markboth{声明及版权}{声明及版权}
 %\addcontentsline{toc}{chapter}{声明}
 \vspace{14pt}
\begin{center}
	{\xiaoerhao {\bf 复旦大学}}\\
	 \vspace{14pt}
	{\xiaoerhao {\bf 学位论文独创性声明}}
\end{center}

\vspace{14pt}

\begin{spacing}{2.0}
%\doublespacing
本人郑重声明:所呈交的学位论文,是本人在导师的指导下,独立进行研究\\
工作所取得的成果。论文中除特别标注的内容外,不包含任何其他个人或机构已\\
经发表或撰写过的研究成果。对本研究做出重要贡献的个人和集体,均已在论文\\
中作了明确的声明并表示了谢意。本声明的法律结果由本人承担。
\end{spacing}
 %本人声明所呈交的学位论文是本人在导师指导下进行的研究工作及取
% 得的研究成果。据我所知,除了文中特别加以标注和致谢的地方外,论文中不包含
% 其他人已经发表或撰写过的研究成果,也不包含为获得~$\underline{\mbox{\kai
%  {复旦大学}}}$~
% 或其他教育机构的学位或证书而使用过的材料。与我一同工作的同志对本研究所
% 做的任何贡献均已在论文中作了明确的说明并表示谢意。
 %\vspace*{8mm}

 {\hfill{
 \begin{tabular}{ll}\\
 作者签名:\underline{\hspace{2.8cm}}& 日期:\underline{\hspace{2cm}}
 \end{tabular}}\par}

 %\vspace*{25mm}
 \vspace{84pt}

\begin{center}
	{\xiaoerhao {\bf 复旦大学}}\\
	 \vspace{14pt}
	{\xiaoerhao {\bf 学位论文使用授权声明}}
\end{center}

\vspace{42pt}

\begin{spacing}{1.5}
本人完全了解复旦大学有关收藏和利用博士、硕士学位论文的规定,即:学\\
校有权收藏、使用并向国家有关部门或机构送交论文的印刷本和电子版本;允许\\
论文被查阅和借阅;学校可以公布论文的全部或部分内容,可以采用影印、缩印\\
或其它复制手段保存论文。涉密学位论文在解密后遵守此规定。
\end{spacing}

% 本学位论文作者完全了解~$\underline{\mbox{\kai
%{复旦大学}}}$~有关保留、
% 使用学位论文的规定,有权保留
% 并向国家有关部门或机构送交论文的复印件和磁盘,允许论文被查阅和借阅。本人授权
% ~$\underline{\mbox{\kai{复旦大学}}}$~
% 可以将学位论文的全部或部分内容编入有关数据库进行检索,可以采用影印、缩印或扫描等
% 复制手段保存、汇编学位论文。
%
% (保密的学位论文在解密后适用本授权书)

 \vspace{14pt}
 {\hfill
 \begin{tabular}{lll}\\
  作者签名:\underline{\hspace{2.8cm}} & 导师签名:\underline{\hspace{2.8cm}} & 日期:\underline{\hspace{2cm}}
 \end{tabular}\par}

 %\vspace*{11mm}


 \begin{center}
 \begin{tabular}{ll}\\
% 学位论文作者毕业后去向:& \\
% 工作单位:复旦大学华山论剑研究所 &
% 电话:+(86)139-1744-9685\\
% 通讯地址:上海市邯郸路220号复旦大学计算机科学技术学院~9999~信箱 & 邮编:310027\\
% E-mail:\href{mailto:cchangyou@gmail.com}{cchangyou@gmail.com} or \href{mailto:072021151@fudan.edu.cn}{072021151@fudan.edu.cn}&

 \end{tabular}
 \end{center}


%\clearpage

%\bibliographystyle{apalike}
%\bibliographystyle{ieeetr}
%\bibliography{reference}
\end{document}
